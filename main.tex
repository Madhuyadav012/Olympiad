\documentclass[12pt,-letter paper]{article}
\usepackage{siunitx}
\usepackage{setspace}
\usepackage{gensymb}
\usepackage{xcolor}
\usepackage{caption}
%\usepackage{subcaption}
\doublespacing
\singlespacing
\usepackage[none]{hyphenat}
\usepackage{amssymb}
\usepackage{relsize}
\usepackage[cmex10]{amsmath}
\usepackage{mathtools}
\usepackage{amsmath}
\usepackage{commath}
\usepackage{amsthm}
\interdisplaylinepenalty=2500
%\savesymbol{iint}
\usepackage{txfonts}
%\restoresymbol{TXF}{iint}
\usepackage{wasysym}
\usepackage{amsthm}
\usepackage{mathrsfs}
\usepackage{txfonts}
\let\vec\mathbf{}
\usepackage{stfloats}
\usepackage{float}
\usepackage{cite}
\usepackage{cases}
\usepackage{subfig}
%\usepackage{xtab}
\usepackage{longtable}
\usepackage{multirow}
%\usepackage{algorithm}
\usepackage{amssymb}
%\usepackage{algpseudocode}
\usepackage{enumitem}
\usepackage{mathtools}
%\usepackage{eenrc}
%\usepackage[framemethod=tikz]{mdframed}
\usepackage{listings}
%\usepackage{listings}
\usepackage[latin1]{inputenc}
%%\usepackage{color}{   
%%\usepackage{lscape}
\usepackage{textcomp}
\usepackage{titling}
\usepackage{hyperref}
%\usepackage{fulbigskip}   
\usepackage{tikz}
\usepackage{graphicx}
\lstset{
  frame=single,
  breaklines=true
}
\let\vec\mathbf{}
\usepackage{enumitem}
\usepackage{graphicx}
\usepackage{siunitx}
\let\vec\mathbf{}
\usepackage{enumitem}
\usepackage{graphicx}
\usepackage{enumitem}
\usepackage{tfrupee}
\usepackage{amsmath}
\usepackage{amssymb}
\usepackage{mwe} % for blindtext and example-image-a in example
\usepackage{wrapfig}
\graphicspath{{figs/}}
\providecommand{\mydet}[1]{\ensuremath{\begin{vmatrix}#1\end{vmatrix}}}
\providecommand{\myvec}[1]{\ensuremath{\begin{bmatrix}#1\end{bmatrix}}}
\providecommand{\cbrak}[1]{\ensuremath{\left\{#1\right\}}}
\providecommand{\brak}[1]{\ensuremath{\left(#1\right)}}

\begin{document}
\begin{enumerate}


		\subsection*{1989  30th IMO Day-1}


\item Prove that the set $\{1,2,.........,1989\}$ can be expressed as the disjoint union of subsets $A_i$\brak{i=1,2,........,117} such that :\\
\brak{i} Each $A_i$ contains $17$ elements ;\\
\brak{ii} The sum of all the elements in each $A_i$ is the same .


\item In an acute-angled triangle $ABC$ the internal bisector of angle $A$ meets the circumcircle of the triangle again at $A_1$. Points $B_1$ and $C_1$ are defined similarly. Let $A_0$ be the point of intersection of the line $AA_1$ with the external bisectors of angles $B$ and $C$. Points $B_0$ and $C_0$ are defined similarly. Prove that:

\brak{i} The area of the triangle $A_0$ $B_0C_0$ is twice the area of the hexagon $AC_1BA_1CB_1$

\brak{ii} The area of the triangle $A_0B_0C_0$ is at least four times the area of the triangle $ABC$.

\item Let $n$ and $k$ be positive integers and let $S$ be a set of $n$ points in the plane such that

\brak{i} No three points of $S$ are collinear, and 

\brak{ii} For any point $P$ of $S$ there are at least $k$ points of $S$ equidistant from $P$.

		Prove that: \begin{align*}k < \frac{1}{2} + \sqrt{2n}.\end{align*}
 \subsection*{1989  30th IMO Day-2}

	\item Let $ABCD$ be a convex quadrilateral such that the sides ${AB, AD, BC}$ satisfy $AB= AD + BC$. There exists a point. $P$ inside the quadrilateral at a distance $h$ from the line $CD$ such that $AP= h+ AD$ and $BP= h + BC$. Show that:\begin{align*}
	\frac{1}{\sqrt{h}}\geq\frac{1}{\sqrt{AD}}+\frac{1}{\sqrt{BC}}\end{align*}.


	\item Prove that for each positive integer $n$ there exist $n$ consecutive positive integers none of which is an integral power of a prime number.

	\item A permutation $\brak{x_1,x_2,....,x_m}$ of the set \{1,2.....,2n\}. where $a$ is a positive integer, is said to have property $P$ if $\mydet{x_i - x_{i+1}} = n $ for at least one in\{1,2,....,2n-1\}. Show that, for each $n$, there are more permitations with property $P$ than without.


  \subsection*{1990  31st IMO Day-1}


	\item Chords $AB$ and $CD$ of a circle imersect at a point $E$ inside the circle. Let $M$ be an interior point of the segment $EB$. The tangent line at $E$ to the circle through $D, E$. and $M$ intersects the lines $BC$ and $AC$ at $F$ and $G$. respectively,
		If \begin{align*}\frac{AM}{AB}=t \end{align*}
			find \begin{align*} \frac{EG}{EF}\end{align*}
		in terms of t .
	\item Let $n_3$ and consider a set $E$ of $2_{n-1}$ distinct points on a circle. Suppose that exactly $k$ of these points are to he colored black. Such a coloring is $"good"$ if there is at least one pair of black points such that the interior of one of the ares between them contains exactly in points from $E$. Find the smallest value of $k$ so that every such coloring of $k$ points of $E$ is good

	\item Determine all integers $n>1$  such that
		\begin{align*} \frac{{2^n}+1}{n^2}\end{align*}\\ is integer.



	\subsection*{1990  31st IMO Day-2}



		\item Let $Q^+$ be the set of positive rational numbers. Construct a function $ f: Q^+ \rightarrow Q^+$ such that 

			\begin{align*}  f\brak{xf\brak{y}}= \frac{f\brak{x}}{y} \end{align*}\\ for all $x , y$ in $Q^+$.

	\item Given an initial integer $n_0 > 1$, two players. $A$ and $B$, choose integers $n_1, n_2 , n_3,.......$ alternately according to the following rules:\\
		Knowing $n_{2k}$, $A$ chooses any integer $n_{2k+2}$ such that \begin{align*} n_{2k}\leq n_{2k+1} \leq n^{2}_2{k} \end{align*}
			Knowing $n_{2k+1}$ , $B$ chooses any integer $n_{2k+2}$ such that \begin{align*}
			\frac{n_{2k+1}}{n_{2k+2}}\end{align*}\\
is a prime raised to a positive integer power.\\
		Plaver $A$ wins the game by choosing the number $1990$: player $B$ wins by choosing the number $1$. For which $n_0$ does:\\
\brak{a}$A$ have a winning strategy?\\
\brak{b} $B$ have a winning strategy?\\
\brak{c} Neither player have a winning strategy?


\item Prove that there exists a convex $1990$-gon with the following two properties\\
\brak{a} All angles are equal.\\
\brak{b} The lengths of the 1990 sides are the numbers $1^2, 2^2, 3^2$,.....,$1990^2$ in some order.

		\subsection*{1991 32nd IMO Day-1}


\item Given a triangle $ABC$, let $I$ be the center of its inscribed circle. The internal bisectors of the angles $A, B, C$ meet the opposite sides in $A', B', C'$ respectively. Prove that
	\begin{align*}\frac{1}{4} < \frac{AI. BI. CI.}{AA'. BB'. CC'.}\leq\frac{8}{27}\end{align*}. 


\item Let $n > 6$ be an integer and $a_1, a_2,....,a_k $ be all the natura numbers less than $n$ and relatively prime to $n$ If \begin{align*}
a_2-a_1=a_3-a_2=......=a_k-a_{k-1} > 0,\end{align*}\\
  prove that $n$ must be either a prime number or a power of 2.


  \item Let $S = \{1,2,3,......,280\}$. Find the smallest integer $n$ such that each $n-$ element subset of $S$ contains five numbers which are pairwise relatively prime.

	  \subsection*{1991 32nd IMO Day-2}

  \item Suppose $G$ is a connected graph with $k$ edges. Prove that it is possible to label the edges $1,2.....k$ in such a way that at each vertex which belongs to two or more edges, the greatest common divisor of the integers labeling those edges is equal to $1$.\\
	  $[$ A graph consists of a set of points, called vertices, together with a set of edges joining certain pairs of distinct vertices. Each pair of vertices. $u, v$ belongs to at most one edge. The graph $G$ is connected if for cach pair of distinct vertices $x, y$ there is some sequence of vertices   $x=v_0,v_1,v_2,.......,v_m = y$  such that each pair $v_i,v_{i+1}\brak{0\leq i < m}$ is joined by an edge of G$.]$

  \item Let $ABC$ be a triangle and $P$ an interior point of $ABC$. Show that at least one of the angles $\angle{PAB}, \angle{PBC}, \angle{PCA}$ is less than or equal to $30\degree$.

  \item An infinite sequence $x_0, x_1, x_2,.... $of real numbers is said to be bounded if there is a constant $C$ such that $\mydet{x_i} \leq C$ for every $i \geq 0$.\\
	  Given any real number $a > 1$, construct a bounded infinite sequence $x_0, x_1, x_2,.....$ Such that\\
	  \begin{align*}\mydet{x_i-x_j}\mydet{i-j}^a \geq 1 \end{align*}\\
	  for every pair of distinct nonnegative integers $i, j$.


\end{enumerate}
\end{document}
